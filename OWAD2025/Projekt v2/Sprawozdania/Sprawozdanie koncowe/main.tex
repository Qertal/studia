\documentclass[a4paper,12pt]{article}
\usepackage[utf8]{inputenc}
\usepackage[T1]{fontenc} 
\usepackage[polish]{babel}
\usepackage{geometry}
\usepackage{hyperref}
\geometry{margin=2.5cm}
\title{Sprawozdanie końcowe — Dashboard COVID-19}
\author{}
\date{}

\begin{document}

\maketitle

\section*{Uczestnicy grupy}
\begin{tabular}{|l|l|}
\hline
Imię i nazwisko & Numer albumu \\
\hline
Kamil Bednarski & 143334 \\
Paweł Drzyzga & 143200 \\
Łukasz Hapeta & 143211 \\
\hline
\end{tabular}

\vspace{1cm}

Z uwagi na zmianę datasetu (poprzedni nie nadawał się na dashboard), wstawiamy nowe pytania do zbioru. Zbiór zawiera dane ze wszystkich krajów na świecie z liczbą zachorowań, zgonów, wyzdrowień oraz aktywnych przypadków podczas epidemii COVID od stycznia 2020 do lipca 2020.

\vspace{0.5cm}
\noindent Link do zbioru: \url{https://www.kaggle.com/datasets/imdevskp/corona-virus-report}

\section*{Opracowane pytania}
\begin{enumerate}
    \item Jakie były kraje z najwyższą śmiertelnością?
    \item Jakie były kraje z najniższą śmiertelnością?
    \item Jak przebiegał rozwój pandemii na świecie od stycznia 2020 do lipca 2020?
    \item Jak przebiegał rozwój pandemii w Europie od stycznia 2020 do lipca 2020?
    \item Jak przebiegał rozwój pandemii w Polsce od stycznia 2020 do lipca 2020?
    \item Rozwój COVID-u w Polsce na tle świata.
    \item Rozwój COVID-u w Polsce na tle Europy.
    \item Liczba zachorowań na milion Polaków.
    \item Liczba zachorowań na milion Europejczyków.
    \item Liczba zachorowań na milion osób na świecie.
\end{enumerate}

\section*{Opis realizacji}

Głównym celem projektu było stworzenie dashboardu, który umożliwia analizę danych dotyczących pandemii COVID-19. Początkowo głównym celem było stworzenie dashboardu z danymi dotyczącymi pandemii COVID-19 w Polsce, jednak w trakcie realizacji projektu zdecydowaliśmy się na rozszerzenie zakresu danych o informacje z całego świata oraz dostosowanie całego dashboardu do analizy danych z różnych krajów.

Aby całkowicie nie rezygnować z początkowego pomysłu, dużo początkowych ustawień jest dostosowanych do Polski, jednak w każdej chwili można zmienić kraj i analizować dane z innych państw, nie tylko pojedyńczo.

Nie da się ukryć, że głównym elementem projektu jest część stworzona w R. Wynika to z tego, że R jest bardzo dobrym narzędziem do analizy danych i wizualizacji. Wykorzystaliśmy bibliotekę Shiny, która pozwala na tworzenie interaktywnych aplikacji webowych w R. Dzięki temu mogliśmy stworzyć dashboard, który umożliwia użytkownikom interakcję z danymi. Dodatkowo, praca w R daje dużo większą swobodę w dostosowywaniu dashboardu do naszych potrzeb, w przeciwieństwie do narzędzi typu Tableau, które są bardziej ograniczone.

\section*{Dashboard w R}

Od ostatniego sprawozdania, w którym został przedstawiony wstępny dashboard, został on znacznie rozbudowany. Dodano wiele nowych funkcji, które umożliwiają lepszą analizę danych. Jest możliwość analizy danych z jednego bądź wielu krajów, porównywanie danego kraju z uśrednionymi wynikami z innych krajów, jak i przeglądanie danych w wersji rankingowej, z możliwością przenoszenia się do szczegółowych informacji z danego kraju.

Wszystkie te funkcje są dostępne w interaktywnym dashboardzie, który można uruchomić lokalnie na komputerze. Wystarczy zainstalować R oraz odpowiednie biblioteki, a następnie uruchomić plik \texttt{apka.R} w środowisku RStudio. Z nazwą pliku wiąże się dziwna sprawa, ponieważ kiedy plik posiadał nazwę \texttt{app.R}, to były problemy z uruchomieniem aplikacji, a konkretnie z uruchomieniem innych plików, zawartych w środowisku, więc zmieniliśmy nazwę na \texttt{apka.R} i problem zniknął.

Od czasu ostatniej prezentacji dashboardu, również została zmodyfikowana część wizualna, która jest teraz bardziej przejrzysta i estetyczna. Pozbyto się znacznej ilości pustych przestrzeni, dodatkowo nie trzeba już nic scrollować w dół, aby zobaczyć wszystkie elementy dashboardu. Wszelkiego rodzaju słupki zawsze są w kolejności malejącej, a sama kolorystyka w całym dashboardzie jest spójna i przyjemna dla oka. 

Z większych zmian, została wyrzucona ostatnia zakładka "Country Details", w której mogliśmy podejrzeć szczegółowe informacje z danego kraju. Została ona zastąpiona przez interakcję z wykresem w części "Country Ranking". Wystarczy kliknąć w słupek odpowiadający danemu krajowi, aby przenieść się do szczegółowych informacji z tego kraju. Jest tam wymienione przede wszystkich "miejsce w rankingu" w konkretnej kategorii, jaka akurat była wybrana na wykresie (confirmed, deaths, recovered, active), a także wykresy, statystyki, tabela z danymi jak i możliwość pobrania tych danych w formacie CSV.

\section*{Dashboard w Tableau}

Dodatkowo został stworzony dashboard w Tableau, który jest bardziej ograniczony, ale również pozwala na analizę danych dotyczących pandemii COVID-19. W naszym przypadku jest on dużo bardziej uboższy, ponieważ ogranicza się głównie do czegoś podobnego, jak pierwszy dashboard w R. Wynika to głównie z ograniczeń narzędzia Tableau, które nie pozwala na tak dużą swobodę w tworzeniu interaktywnych aplikacji webowych jak R. Stworzenie dashboardu w Tableau, który by odpowiadał naszemu pomysłowi, zajęłoby dużo więcej czasu i wymagałoby większej ilości pracy, dlatego zdecydowaliśmy się na stworzenie prostszego dashboardu, a skupiliśmy się na dashboardzie w R.

Link do dashboardu w Tableau Public \url{https://public.tableau.com/app/profile/pawe.drzyzga/viz/OWAD2025Dashboard/Dashboard1}

\section*{Porównanie prac}

Praca w R i Tableau różni się głównie podejściem do analizy danych. R jest językiem programowania, który pozwala na dużą swobodę w tworzeniu interaktywnych aplikacji webowych, natomiast Tableau jest narzędziem do wizualizacji danych, które jest bardziej ograniczone. W R możemy stworzyć aplikację, która będzie dostosowana do naszych potrzeb, natomiast w Tableau musimy się dostosować do narzędzia. Z jednej strony, Tableau jest bardziej przyjazne dla użytkownika, ponieważ nie wymaga znajomości programowania, pozwala na szybkie tworzenie wizualizacji danych i jest bardziej intuicyjne. Z drugiej strony, R daje nam większą swobodę w tworzeniu interaktywnych aplikacji webowych, które mogą być dostosowane do naszych potrzeb. W przypadku R tak naprawdę ogranicza nas tylko wyobraźnia, a w przypadku Tableau jesteśmy ograniczeni przez narzędzie. 

Używanie Tableau w naszym przypadku nie miało większego sensu, ponieważ nie było możliwości stworzenia dashboardu, który by odpowiadał naszym potrzebom. Stawianie sobie takich ograniczeń nie miało sensu, kiedy jesteśmy w stanie stworzyć coś duzo lepszego i bardziej funkcjonalnego czy kreatywniejszego w R. Stąd uważamy, że praca w R, choć na początku może wydawać się trudniejsza czy bardziej wymagająca, jest dużo bardziej opłacalna w dłuższej perspektywie czasowej.

\section*{Historia}

Pierwsze uruchomienie dashboardu prowadzi nas do pierwszego panelu pod nazwą "Global Trends", z automatycznie wybranym krajem "Polska" oraz pełnym zasięgiem dat, jaki był dostępny w zbiorze danych. Możemy tu zobaczyć wykresy przedstawiające rozwój pandemii w Polsce, sprwadzić jak niektóre dedyzje np rządowe wpłynęły na rozwój pandemii. Posiadamy statystyki jak na dany (ostatni) dzień wygląda sytuacja, mamy również histogramy, jak i wykres kołowy, który dobrze obrazuje udział poszczególnych grup w całosci potwierdzonych przypadków COVID-19 w Polsce. Dużą zaletą tej sekcji jest to, że nie musimy się ograniczać tylko do Polski, możemy zmienić kraj na dowolny inny i zobaczyć jak wygląda sytuacja w danym kraju. Co więcej, możemy wybrać kilka krajów, jakis konkretny kontynent, czy nawet cały świat i zobaczyć jak wygląda sytuacja w danym regionie.

Kolejną sekcją jest "Country Comparison", która pozwala na porównanie wybranych krajów pod względem liczby zachorowań, zgonów, wyzdrowień i aktywnych przypadków. Możemy tu zobaczyć, jak dany kraj wypada na tle innych krajów, czy to wybranych pojedyńczo, czy też uśrednionych wyników z dowolnej liczby innych krajów. Jest to bardzo przydatne narzędzie, które pozwala na szybką analizę danych i porównanie sytuacji w różnych krajach. Szczególnie, że oprócz samych wykresów, mamy również tabelę z danymi, która pozwala na dokładniejsze przeanalizowanie sytuacji w danym kraju. Wykresy liniowe zostały rozdzielone na 4, aby nie było problemu z czytelnością. Tutaj szczególnie jest przydatny wykres kołowy, który pozwala na szybkie zobaczenie udziału poszczególnych krajów w całkowitej liczbie zachorowań, zgonów, wyzdrowień i aktywnych przypadków. Oczywiście tutaj tak samo jak wcześniej, możemy wybrać sobie zakres dat jaki tylko potrzebujemy.

Trzecią sekcją jest "Country Ranking", która pozwala na przeglądanie krajów w formie rankingu, według wybranej kategorii (liczba zachorowań, zgonów, wyzdrowień, aktywnych przypadków). Możemy tu zobaczyć, które kraje mają najwięcej zachorowań, zgonów, wyzdrowień i aktywnych przypadków. Możemy sobie sortować cały wykres czy to rosnące, czy to malejąco, tak jak wybrać liczbę wyświetlanych krajów z przedziału [1,25]. Dodatkowo, klikając na odpowiedni słupek (odpowiadajy danemu państwu) zostaniemy przeniesieni do sekcji, która pokazuje bardziej sczególowe informacje o danym kraju, takie jak jego statystyki, interaktywny wykres liniowy do głębszej analizy, histogram, tabelka z danymi, a co ciekawsze, możliwość pobrania całej tej tabelki do pliku w formacie csv. Dodatkowo na samej górze jest wyświetlana nazwa kraju, rodzaj przypadku, który mieliśmy wybrany w Country Ranking, jak i miejsce tego kraju w tym własnie rankingu. Możemy oczywiście wrócic do poprzedniego panelu korzystając z przeznaczonego do tego przycisku, a następnie wybrać tam kraj referencyny. Wybranie takiego kraju sprawi, że zostanie on wyróżniony, a dodatkowo na wykresie zobaczymy kraje "sąsiadujące" z nim, jeśli chodzi o wynik w danej kategorii. Interakcja działa tutaj w identyczny sposób.

\section*{Wnioski końcowe}

Projekt poświęcony analizie danych dotyczących pandemii COVID-19 zaowocował stworzeniem dwóch dashboardów — w R i Tableau — z czego główny nacisk został położony na rozwiązanie oparte na środowisku R. Dzięki wykorzystaniu bibliotek takich jak `shiny`, `ggplot2` czy `plotly`, udało się stworzyć w pełni interaktywną aplikację webową, która pozwala użytkownikowi na dynamiczną analizę danych, porównywanie krajów, przeglądanie rankingów, a także dostęp do szczegółowych informacji o wybranym państwie.

Stworzony dashboard w R pozwala w przejrzysty i intuicyjny sposób odpowiadać na wszystkie postawione na początku pytania badawcze, takie jak porównania zachorowań i zgonów w ujęciu globalnym, kontynentalnym czy krajowym, a także umożliwia znacznie szersze analizy, nieujęte bezpośrednio w zestawie początkowych problemów. Elastyczność oraz możliwości dostosowywania interfejsu sprawiają, że narzędzie to może być z powodzeniem wykorzystywane także w innych kontekstach analitycznych.

Dashboard w Tableau, mimo że również zawiera podstawowe informacje o przebiegu pandemii, pełni w tym przypadku funkcję pomocniczą. Ograniczenia platformy Tableau w zakresie interaktywności i swobodnego modelowania logiki działania aplikacji sprawiły, że to środowisko R okazało się znacznie bardziej adekwatne do założonych celów projektu.


\end{document}
