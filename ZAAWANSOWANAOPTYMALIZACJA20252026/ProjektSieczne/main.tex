% filepath: c:\REPO\studia\ZAAWANSOWANAOPTYMALIZACJA20252026\ProjektSieczne\main.tex
\documentclass[10pt,a4paper]{article}

% ...existing code...
% Kodowanie i język
\usepackage[utf8]{inputenc}
\usepackage[T1]{fontenc}
\usepackage[polish]{babel}

% Układ strony i mikrotypografia
\usepackage[a4paper,margin=15mm]{geometry}
\usepackage{lmodern}
\usepackage{microtype}

% Podstawowe pakiety
\usepackage{graphicx}
\usepackage{amsmath,amssymb}
\usepackage{siunitx}
\usepackage{booktabs}
\usepackage{caption}
\usepackage{hyperref}
\usepackage{algorithm}
\usepackage{algpseudocode}
\usepackage{float}
\usepackage{longtable}
\usepackage{booktabs}
\usepackage{minted}
\usepackage{listings}
\usepackage{xcolor}
\lstset{
  language=Matlab,
  basicstyle=\ttfamily\small,
  keywordstyle=\color{blue},
  commentstyle=\color{gray},
  frame=single,
  breaklines=true,
  numbers=left,
  numberstyle=\tiny,
  xleftmargin=1em
}


\hypersetup{colorlinks=true,linkcolor=blue,citecolor=blue,urlcolor=blue}

% Bibliografia (użyj biber lub zmień na natbib jeśli wolisz)
\usepackage[backend=biber,style=numeric,sorting=none]{biblatex}
\addbibresource{references.bib} % utwórz plik references.bib w katalogu projektu

% ...existing code...
\title{Metoda Siecznych}
\author{Paweł Drzyzga \\ Wydział Informatyki i Matematyki \\ Politechnika Krakowska}
\date{\today}

\begin{document}

\maketitle

% \begin{abstract}
% Krótki opis pracy: cel, metoda, najważniejsze wyniki. Tutaj umieść streszczenie po polsku.
% \end{abstract}

\tableofcontents
\clearpage

\section{Opis teoretyczny}
\subsection{Opis teoretyczny metody -- Metoda siecznych}

\subsubsection{Charakterystyka metody}

Metoda siecznych, zwana czasem również „metodą cięciw”, to algorytm numeryczny służący do znajdowania miejsc zerowych funkcji jednej zmiennej, czyli rozwiązywania równania
\[
f(x) = 0.
\]
W literaturze polskojęzycznej metoda najczęściej opisana jest jako algorytm interpolacji liniowej, który nie wymaga znajomości pochodnej funkcji ani nawet założenia różniczkowalności. 

\subsubsection{Idea działania}

Zakładamy, że funkcja $f(x)$ jest ciągła (przynajmniej na małym odcinku), a na przedziale $[a,b]$ przyjmuje wartości o różnych znakach (równoważnie: $f(a)f(b)<0$). Wówczas istnieje przynajmniej jeden pierwiastek w tym przedziale. Metoda siecznych polega na zastąpieniu krzywej $y=f(x)$ na odcinku przybliżoną prostą (sieczną) przebiegającą przez punkty $(x_{n-1},f(x_{n-1}))$ i $(x_n,f(x_n))$. Przecięcie tej siecznej z osią $x$ daje nowe przybliżenie $x_{n+1}$.

\subsubsection{Wzór rekurencyjny}

W podstawowej postaci możemy zapisać system początkowy i wzór:
\[
\begin{cases}
x_0 = a,\\[4pt]
x_1 = b,\\[4pt]
x_{n+1} = \dfrac{f(x_{n})\,x_{n-1} - f(x_{n-1})\,x_{n}}{f(x_{n}) - f(x_{n-1})}.
\end{cases}
\]

Alternatywna, często stosowana postać (również poprawna) to:
\[
x_{n+1} = x_n - f(x_n)\,\frac{x_n - x_{n-1}}{f(x_n) - f(x_{n-1})}.
\]

\subsubsection{Wymagania i zbieżność}

Aby metoda miała szansę działać prawidłowo, typowo wymagamy między innymi:
\begin{itemize}
  \item funkcja $f$ jest ciągła na przedziale $[a,b]$,
  \item wartości na końcach mają różne znaki (np. $f(a)f(b)<0$), co zapewnia istnienie pierwiastka.
\end{itemize}
% Metoda nie zawsze jest zbieżna — zdarzają się sytuacje, gdy kolejne przybliżenia nie zbliżają się do pierwiastka lub „skaczą”.

\section{Schemat blokowy}

\begin{figure}[H]
  \centering
  \includegraphics[width=0.6\textwidth]{SchematBlokowy.png}
  \caption{Schemat blokowy metody siecznych.}
  \label{fig:schemat_blokowy_sieczne}
\end{figure}



\section{Pseudokod algorytmu}
\begin{algorithm}[H]
\caption{Metoda siecznych dla wyznaczania ekstremum funkcji jednej zmiennej}
\label{alg:sieczne}
\begin{algorithmic}[1]
\Require Funkcja $f(x)$, przybliżenia początkowe $x_0, x_1$, tolerancja $\epsilon$, maksymalna liczba iteracji $maxit$, krok różniczkowania $h$
\State Oblicz pochodną numeryczną: $g(x) = \dfrac{f(x+h) - f(x-h)}{2h}$ (jeśli pochodna analityczna nie jest dostępna)
\State $g_0 \gets g(x_0)$, \quad $g_1 \gets g(x_1)$
\State $iter \gets 0$
\While{$iter < maxit$}
    \If{$g_1 - g_0 = 0$}
        \State \textbf{przerwij} \Comment{dzielnik równy zero}
    \EndIf
    \State $x_2 \gets x_1 - g_1 \cdot \dfrac{x_1 - x_0}{g_1 - g_0}$
    \If{$|x_2 - x_1| < \epsilon$}
        \State \Return $x_{min} = x_2$, $f_{min} = f(x_2)$, liczba iteracji = $iter$
    \EndIf
    \State $x_0 \gets x_1$
    \State $g_0 \gets g_1$
    \State $x_1 \gets x_2$
    \State $g_1 \gets g(x_1)$
    \State $iter \gets iter + 1$
\EndWhile
\State \Return komunikat: \textit{"Osiągnięto maksymalną liczbę iteracji"}
\end{algorithmic}
\end{algorithm}

\section{Implementacja w Matlab}
% \begin{lstlisting}[caption={Funkcja \texttt{secant\_extremum.m} -- metoda siecznych dla $f'(x)=0$},label={lst:secant}]

\begin{minted}[frame=single,fontsize=\small,linenos]{matlab}
function [xmin, fmin, iter, hist] = secant_extremum(f, x0, x1, tol, maxit, h)

    if nargin < 4 || isempty(tol),   tol   = 1e-8; end
    if nargin < 5 || isempty(maxit), maxit = 100;  end
    if nargin < 6 || isempty(h),     h     = 1e-6; end

    % Funkcja pomocnicza: numeryczna pochodna centralna
    g = @(x) (f(x + h) - f(x - h)) / (2*h);

    xkm1 = x0;
    xk   = x1;
    gkm1 = g(xkm1);
    gk   = g(xk);

    hist = [0, xkm1, gkm1; 1, xk, gk];

    for k = 1:maxit
        denom = (gk - gkm1);
        if denom == 0
            warning('SECANT_EXTREMUM:ZeroDenominator', 'Dzielnik równy zero — przerywam.');
            break;
        end
        xkp1 = xk - gk * (xk - xkm1) / denom;

        hist = [hist; k+1, xkp1, g(xkp1)]; %#ok<AGROW>

        if abs(xkp1 - xk) < tol
            xk = xkp1;
            break;
        end

        xkm1 = xk;   gkm1 = gk;
        xk   = xkp1; gk   = g(xk);
    end

    xmin = xk;
    fmin = f(xmin);
    iter = size(hist,1)-1;
end
\end{minted}

\section{Prosty przykład liczbowy}
\subsection*{Przykład: metoda siecznych dla $g(x)=x^2-2$ na przedziale $[1,2]$}

Dane początkowe: $x_0 = 1$, $x_1 = 2$, $g(x) = x^2 - 2$.  
Iteracyjny wzór metody siecznych:
\[
x_{k+1} = x_k - g(x_k)\,\frac{x_k - x_{k-1}}{g(x_k) - g(x_{k-1})}.
\]

\noindent\textbf{Iteracja 1.}
\[
g(x_0) = 1^2 - 2 = -1, \qquad g(x_1) = 2^2 - 2 = 2.
\]
\[
x_2 = 2 - 2\cdot\frac{2 - 1}{\,2 - (-1)\,} 
     = 2 - \frac{2}{3} 
     = \frac{4}{3}, 
\qquad 
g\!\left(\tfrac{4}{3}\right) = \tfrac{16}{9} - 2 = -\tfrac{2}{9} \approx -0.222222.
\]

\noindent\textbf{Iteracja 2.}
\[
x_3 
= \frac{4}{3} - \!\left(-\tfrac{2}{9}\right)
  \frac{\,\tfrac{4}{3} - 2\,}{-\tfrac{2}{9} - 2}
= \frac{4}{3} - \Big(-\tfrac{2}{9}\Big)\cdot\frac{-\tfrac{2}{3}}{-\tfrac{20}{9}}
= \frac{4}{3} + \frac{1}{15}
= \frac{7}{5},
\]
\[
g\!\left(\tfrac{7}{5}\right) = \tfrac{49}{25} - 2 = -\tfrac{1}{25} = -0.04.
\]

\noindent\textbf{Iteracja 3.}
\[
x_4 
= \frac{7}{5} - \!\left(-\tfrac{1}{25}\right)
  \frac{\,\tfrac{7}{5} - \tfrac{4}{3}\,}{-\tfrac{1}{25} - \left(-\tfrac{2}{9}\right)}
= \frac{7}{5} - \Big(-\tfrac{1}{25}\Big)\cdot
  \frac{\tfrac{1}{15}}{\tfrac{41}{225}}
= \frac{7}{5} + \frac{3}{205}
= \frac{58}{41},
\]
\[
g\!\left(\tfrac{58}{41}\right) 
= \left(\tfrac{58}{41}\right)^{\!2} - 2 
= \tfrac{3364}{1681} - 2 
= \tfrac{2}{1681} \approx 0.00118977.
\]

\noindent\textbf{Iteracja 4.}
\[
x_5 
= x_4 - g(x_4)\,\frac{x_4 - x_3}{g(x_4) - g(x_3)}
= \tfrac{58}{41} - \tfrac{2}{1681}\cdot \frac{\,\tfrac{58}{41} - \tfrac{7}{5}\,}{\,\tfrac{2}{1681} - \left(-\tfrac{1}{25}\right)}
= \tfrac{33\,456}{23\,657},
\]
\[
g\!\left(\tfrac{33\,456}{23\,657}\right) 
= \left(\tfrac{33\,456}{23\,657}\right)^{\!2} - 2 
= \frac{-3362}{23\,657^{2}} 
\approx -6.01\times 10^{-6}.
\]

\[
\boxed{
x_2 = \tfrac{4}{3},\quad 
x_3 = \tfrac{7}{5},\quad 
x_4 = \tfrac{58}{41},\quad
x_5 = \tfrac{33\,456}{23\,657} \approx 1.4142 \approx \sqrt{2}.
}
\]

\noindent
Dla przejrzystości zestawmy wszystkie wartości $x_n$ i $g(x_n)$:

\begin{center}
\begin{tabular}{ccccc}
\toprule
$n$ & $x_n$ (dokładnie) & $x_n$ ($\approx$) & $g(x_n)$ (dokładnie) & $g(x_n)$ ($\approx$) \\
\midrule
0 & $1$ & $1.000000$ & $-1$ & $-1.000000$ \\
1 & $2$ & $2.000000$ & $2$ & $2.000000$ \\
2 & $\tfrac{4}{3}$ & $1.333333$ & $-\tfrac{2}{9}$ & $-0.222222$ \\
3 & $\tfrac{7}{5}$ & $1.400000$ & $-\tfrac{1}{25}$ & $-0.040000$ \\
4 & $\tfrac{58}{41}$ & $1.414634$ & $\tfrac{2}{1681}$ & $0.001190$ \\
5 & $\tfrac{33\,456}{23\,657}$ & $1.414200$ & $-\tfrac{3362}{23\,657^{2}}$ & $-6.01\times 10^{-6}$ \\
\bottomrule
\end{tabular}
\end{center}

\noindent
Widzimy, że kolejne przybliżenia $x_n$ szybko zbliżają się do $\sqrt{2}\approx1.41421356$, 
a wartości $g(x_n)$ zmieniają znak naprzemiennie, co jest charakterystyczne 
dla zbieżności metody siecznych w pobliżu pierwiastka.



% \end{lstlisting}

\section{Porównanie}
\begin{longtable}{l r r r r r r r r}
\caption{Porównanie wyników metody siecznych (dla $f'(x)=0$) i \texttt{fminbnd} (nowy zestaw 15 funkcji).}
\label{tab:porownanie_sieczne_new}\\
\toprule
                  funkcja $f(x)$ &  $a$ & $b$ & $x_{\min}^{\mathrm{sec}}$ & $f(x_{\min}^{\mathrm{sec}})$ & $it_{\mathrm{sec}}$ & $x_{\min}^{\mathrm{fminbnd}}$ & $f(x_{\min}^{\mathrm{fminbnd}})$ & $it_{\mathrm{fminbnd}}$ \\
\midrule
\endfirsthead
\caption[]{Porównanie wyników metody siecznych (dla $f'(x)=0$) i \texttt{fminbnd} (nowy zestaw 15 funkcji).} \\
\toprule
                  funkcja $f(x)$ &  $a$ & $b$ & $x_{\min}^{\mathrm{sec}}$ & $f(x_{\min}^{\mathrm{sec}})$ & $it_{\mathrm{sec}}$ & $x_{\min}^{\mathrm{fminbnd}}$ & $f(x_{\min}^{\mathrm{fminbnd}})$ & $it_{\mathrm{fminbnd}}$ \\
\midrule
\endhead
\midrule
\multicolumn{9}{r}{{Continued on next page}} \\
\midrule
\endfoot

\bottomrule
\endlastfoot
      $(x-2)^{2} + 0.2*\sin(5x)$ &   -3 &   5 &                   2.14166 &                    -0.171736 &                  11 &                       2.14166 &                        -0.171736 &                      12 \\
                  $\exp(x) - 3x$ &   -2 &   2 &                   1.09861 &                    -0.295837 &                   7 &                       1.09861 &                        -0.295837 &                      11 \\
                 $(x-1)^{2} + 1$ &   -1 &   3 &                         1 &                            1 &                   3 &                             1 &                                1 &                       5 \\
               $\cos(x) + x^{2}$ &   -3 &   3 &                         0 &                            1 &                   3 &                    1.1436e-08 &                                1 &                      26 \\
              $(x+3)^{2} + 0.1x$ &   -4 &  -2 &                     -3.05 &                      -0.3025 &                   3 &                         -3.05 &                          -0.3025 &                       5 \\
          $(x+1)^{2} + \exp(-x)$ &   -1 &   0 &                 -0.314923 &                      1.83948 &                   6 &                     -0.314923 &                          1.83948 &                       8 \\
             $(x-0.5)^{2} + 0.2$ &   -1 &   2 &                       0.5 &                          0.2 &                   3 &                           0.5 &                              0.2 &                       5 \\
          $x*\sin(x) + x^{2}/10$ & -0.5 & 0.5 &                         0 &                            0 &                   3 &                  -1.38778e-17 &                      2.11852e-34 &                       5 \\
              $\tanh(x) + x^{2}$ &   -1 &   0 &                 -0.420916 &                    -0.220532 &                   6 &                     -0.420916 &                        -0.220532 &                       9 \\
           $(x-5)^{2} + \sin(x)$ &  4.9 & 5.1 &                   4.90452 &                    -0.972483 &                   5 &                       4.90452 &                        -0.972483 &                       9 \\
           $x^{2} + 0.1*\sin(x)$ &   -3 &   3 &                -0.0499377 &                  -0.00249792 &                   6 &                    -0.0499377 &                      -0.00249792 &                       9 \\
$\log(1+\exp(x)) + (x-2)^{2}/10$ &   -1 &   0 &                 -0.222731 &                      1.08202 &                   5 &                     -0.222731 &                          1.08202 &                      12 \\
              $x^{2} + \exp(-x)$ &   -2 &   2 &                  0.351734 &                     0.827184 &                   7 &                      0.351734 &                         0.827184 &                      10 \\
                   $(x+0.2)^{2}$ &   -1 &   1 &                      -0.2 &                            0 &                   3 &                          -0.2 &                                0 &                       5 \\
      $(x+0.5)^{4} + \cos(3x)/3$ &   -1 &   0 &                   -0.9352 &                    -0.278823 &                  18 &                       -0.9352 &                        -0.278823 &                      11 \\
\end{longtable}


\section{Podsumowanie i wnioski}

W pracy przeprowadzono eksperyment badawczy nad metodą siecznych w dwóch zastosowaniach:
(i) wyznaczanie pierwiastka równania $g(x)=0$ (klasyczna postać), oraz
(ii) wyznaczanie ekstremum funkcji jednej zmiennej przez rozwiązywanie równania $f'(x)=0$ z pochodną aproksymowaną różnicą centralną.
Przygotowano opis teoretyczny, schemat blokowy, implementację oraz przykład liczbowy, a w sekcji \emph{Porównanie} zestawiono wyniki jakościowe i ilościowe.

Najważniejsze obserwacje z eksperymentu są następujące:
\begin{itemize}
  \item \textbf{Zbieżność.} Zbiezność w tym algorytmie występuje bardzo szybko. W praktyce przekłada się to na szybki spadek błędu, co dobrze ilustruje przykład wyznaczania $\sqrt{2}$.
  \item \textbf{Koszt obliczeniowy.} W wariancie korzeniowym jedna iteracja wymaga jednej oceny funkcji. W wariancie ekstremum (z $f'(x)\approx\tfrac{f(x+h)-f(x-h)}{2h}$) potrzeba dwóch ocen $f$ na krok.
  \item \textbf{Wrażliwość na punkty startowe.} Skuteczność i szybkość zbieżności silnie zależą od doboru $x_0, x_1$. Gdy wartości pochodnej/\-funkcji w punktach startowych są bardzo zbliżone, pojawia się niestabilność (mianownik bliski zeru) i skoki przybliżeń — konieczne są zabezpieczenia w kodzie.
  \item \textbf{Kryteria stopu.} Stosowanie łącznego kryterium $|x_{k+1}-x_k|<\varepsilon$ \emph{i} $|f(x_k)|<\varepsilon_f$ jest praktycznie skuteczniejsze niż samo kryterium w przestrzeni $x$.
\end{itemize}

\paragraph{Wnioski praktyczne.}
\begin{itemize}
  \item \emph{Rekomendacja użycia:} metoda siecznych jest dobrym wyborem, gdy pochodna nie jest dostępna, a zależy nam na szybkości. Sprawdza się też jako szybkie ,,dowykańczanie'' po wstępnym zawężeniu przedziału.
  \item \emph{Inicjalizacja:} warto zaczynać z pary $(x_0,x_1)$ uzyskanej metodą przedziałową zapewniającą $f(x_0)f(x_1)<0$ (dla problemu korzeniowego). Zwiększa to szanse zbieżności i stabilność iteracji.
\end{itemize}

Podsumowując, metoda siecznych jest atrakcyjnym kompromisem między prostotą implementacji, kosztem obliczeń i szybkością zbieżności. Przy właściwej inicjalizacji i z rozsądnymi zabezpieczeniami stanowi praktyczne narzędzie zarówno do wyznaczania pierwiastków, jak i ekstremów funkcji jednej zmiennej.

\end{document}